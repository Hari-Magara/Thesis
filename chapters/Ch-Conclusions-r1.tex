\section{Summary}

Complex LSF wall systems are an integral part of cold-formed steel construction, which includes the double stud, shaftliner and staggered stud LSF wall systems. However, past research studies conducted throughout the globe have always been limited to the investigation of conventional single stud LSF wall systems. Although, extensive research has been conducted on single stud LSF walls systems, there has been very limited research data available on the above-mentioned complex LSF wall systems. Therefore, this research investigated the fire performance of the complex LSF wall systems through experimental and numerical studies. Their fire resistance levels (FRL) were determined based on full-scale fire tests and advanced numerical analysis, thereby facilitating the usage of these complex LSF walls in cold-formed steel construction. In particular, the staggered stud wall systems which were preferably used under non-load bearing conditions, were tested under full-scale load bearing conditions with omega noggings and end cleat connections, to enable their use as load bearing walls. Numerical thermal models were developed using a CFD program, FDS, in predicting the thermal behaviour of the complex LSF wall systems in fire. The sequentially coupled-temperature displacement structural FE models were then developed and used to predict the failure times of these LSF wall systems by incorporating the time-temperature inputs from the FDS thermal analysis. This research has yielded important experimental and numerical results, which can be used by engineers and designers in safely and cost-effectively designing the three types of complex LSF wall systems for various applications using cold-formed steel construction.

\section{Research Outcomes}

The aim of this research were detailed in \Cref{sec:research-aim} of \Cref{ch:Introduction}. The aims achieved as the outcome of this research are detailed next.

\subsection{Selection of Complex LSF wall Systems}

Conducting full-scale fire tests are cumbersome, time-consuming and expensive. Therefore, the initial literature review conducted in \Cref{ch:Literature} not only investigated the research studies conducted on the fire performance of LSF wall systems but also investigated the availability and suitability of the most commonly used complex LSF walls systems. The ground work conducted as a part of this research included leading expert opinions in the cold-formed steel industry in Australia and research consideration of the wall configurations were narrowed down to double stud, shaftliner and staggered stud LSF wall systems. As determination of fire performance was the motto of this research fire rated plasterboards were only used for this research. Also, the steel stud sections used for this research were locally procured to better understand the behaviour of the LSF walls currently used in the cold-formed steel industry. Unlike past research in which the steel stud sections were folded in-house through press break, the studs used in this research were procured from a roll-form manufacturer. This ensured the frames used for experimental investigations were precise to the nearest mm. Also the pre-punch holes in the studs and tracks for screw fastening facilitated easier assembly of the wall frame during construction of the test wall.  

\subsection{Ambient Temperature Structural Performance of Complex LSF Walls}
\begin{itemize}
	\item Full-scale ambient temperature axial compression capacity tests were conducted on double stud and staggered stud LSF wall systems. The behaviour of the studs under axial compression and the ultimate compression capacities of the test wall were determined. Local buckling of studs were the predominant failure mode in both the LSF wall configurations.
	\item The effective plasterboard restraints provided to the studs in combination with the nogging restraints were capable to provide the required lateral restraints to the stud preventing minor axis (in-plane) buckling in the test wall. In the case of staggered stud LSF wall system, the provision of omega nogging along with the end cleats facilitated the staggered stud wall system to withstand axial compression load similar to a double stud LSF wall system, making them suitable for load bearing applications under fire.
	\item Experimental investigations revealed that strengthening the edge studs in a test wall was necessary to avoid bearing failures and to accurately determine the axial compression capacity of the test wall. Nesting the edge studs was found to be the better alternative under ambient capacity tests.  
	\item Axial compression capacity predictions using the new AS 4600 design equations based on Direct Strength Method (DSM), was able to predict the capacities of the test wall with reasonable accuracy.  
\end{itemize}

\subsection{Fire Performance of Complex LSF Walls}

Ten full-scale fire tests on complex LSF walls were conducted for this research. Detailed analyses of the fire test results of double stud walls were undertaken by comparing them against the fire test results of conventional single stud LSF walls. The differences in heat transfer mechanism between the single and double stud walls were clearly identified. Following conclusions are drawn from the conducted full-scale fire tests.
\begin{itemize}
	\item The combined presence of wider cavity and discontinuous stud rows was identified as the reason for the observed delayed heat transfer mechanism in double stud walls. Comparison with a single stud wall of wider cavity proves that the existence of wider cavity alone does not influence the delayed heat transfer mechanism. The presence of a discontinuous stud arrangement is the main contributor to the delayed heat transfer.
	\item The presence of thinner studs did not affected the unique heat transfer mechanism in double stud walls as the plateau region in the time-temperature curve was clearly visible in non-cavity insulated double stud wall fire tests.
	\item Comparison of the time-temperature curves of double stud walls under different load ratios showed that the heat transfer mechanism was unaffected by the higher load ratio.
	\item Despite the difference in lateral restraints to the studs due to the discontinuous rows of studs, the fire test results showed that the failure time of double stud walls was higher than that of similar single stud walls under the same load ratio. This shows that the discontinuous stud arrangement does not affect the structural performance of double stud walls in fire.
\end{itemize}

The conducted fire tests also included two full-scale fire tests on non-load bearing LSF walls with varying cavity depth. Shaftliner and staggered stud LSF walls with varying cavity depth were also tested under non-load bearing conditions. The following specific conclusions are discussed next.
\begin{itemize}
	\item Increasing the cavity depth of LSF walls improved the fire performance significantly. 
	\item The temperature distribution across the wall depth decreased with increasing cavity depth. This is due to the presence of air movement within the wider cavity resulting in natural convection. Radiation from the plasterboards and studs into the cavity also contributed to this behaviour. This is evident from the unique plateau region observed in the plasterboard time-temperature curves of staggered stud wall. This plateau region is also due to the discontinuous stud arrangement in staggered stud wall. However, this was not experienced in shaftliner wall as the effective cavity depth was 90 mm. Continuous stud arrangement with a middle plasterboard layer was identified as the reason behind the absence of plateau region in the time-temperature curves of shaftliner wall tested in this study.
	\item Regardless of the cavity depth, plasterboard fall-off also significantly governed the fire resistance of single plasterboard lined walls as it exposed the thin-walled steel studs and the remaining plasterboard layers to higher temperatures. 
	\item In LSF walls with double plasterboard linings, plasterboard fall-off did not significantly affect the fire resistance. The calcination process of the fire side plasterboards was delayed in this situation thereby resulting in lower temperature gradient across the wall depth. The lower ambient side temperatures allowed the ambient side plasterboards to remain intact till the end of the fire test. 
	\item In the staggered stud wall fire test, the absence of lateral restraints to both flanges of the studs resulted in the structural failure of studs, whereas in the shaftliner wall fire test with a middle plasterboard layer, the presence of effective plasterboard restraints increased the fire resistance. These results show the importance of plasterboard restraints to both flanges of the thin-walled steel studs in LSF walls.
	\item Although increasing the cavity depth resulted in higher fire resistance, splitting the cavity of shaftliner LSF wall with a middle plasterboard layer was found to be more beneficial in terms of fire resistance. In the fire test of shaftliner LSF wall, lower ambient side temperatures were observed in comparison with the fire test of staggered stud LSF wall. Shaftliner wall arrangement can eliminate the structural inadequacy failures of steel studs and provide higher fire resistance under non-load bearing conditions. 
	\item As the unique thermal behaviour of the double stud, shaftliner and staggered stud walls are adequately identified including their overall fire performance, these walls can be used in the construction industry where single stud LSF walls could not satisfy the FRL and acoustic requirements.
	\item The fire test data from this research were also used to develop and validate numerical models to predict the fire performance of double stud LSF walls.  
\end{itemize}

Considering the importance of cavity insulation in LSF walls for sound insulation, this research study has also investigated the response of cavity insulated double stud and staggered stud LSF walls in fire. Three full-scale fire tests out of the ten were conducted with cavity insulation for this research study and valuable time-temperature data for the cavity insulated double and staggered stud LSF walls were determined. The corresponding conclusions are discussed next. 
\begin{itemize}
	\item Despite the presence of wider cavity depth in double stud LSF walls, the presence of cavity insulation entraps the heat on the fire exposed side of the double stud LSF wall. This causes the heat to accumulate on the fire side sides only, thereby increasing the corresponding hot flange temperatures significantly. This is a detrimental effect if the cavity insulation is provided on both the studs rows in double stud wall Tests-T5 and T6.
	\item In Test-T7 the cavity insulation was provided in the ambient stud rows only to allow the heat within the cavity thereby moving the cavity insulation away from the fire exposed side. However, this did not provide beneficial results as the heat was still entrapped on the fire side studs thereby resulting in a similar time-temperature profile in comparison with Tests-T5 and T6. 
	\item Comparison with a non-cavity insulated LSF wall revealed that the plateau region in the time-temperature curves on the plasterboard and studs are not evident in the cavity insulated LSF walls. The constructive effect provided by the discontinuous stud arrangement and the wider cavity did not result in delayed heat transfer mechanism experienced in non-cavity insulated double stud LSF walls.
	\item Plasterboard open up in the fire tests of cavity insulated double and staggered stud LSF walls was found to be the major contributor in sudden temperature increase of the studs under load bearing conditions. Also, the additional pressure developed within the cavity due to the temperature entrapment may have influenced the plasterboard open up and needs further detailed investigation. 
	\item Buckling of the fire side row studs was observed in the cavity insulated staggered stud wall Test-T10 further affirming the claim of heat entrapment in cavity insulated complex LSF walls. This shows that the heat entrapment is a major contributor in premature failure of the steel studs irrespective of the wall configuration in complex LSF walls and needs further investigation.
\end{itemize}

\subsection{Thermal modelling of Complex LSF Walls}
\begin{itemize}
	\item Detailed investigations were conducted to develop suitable thermal numerical models and to investigate the fire performance of the complex LSF wall configurations. 
	\item Initial investigation on the developed thermal models in SAFIR software package resulted in conservative time-temperature predictions of the conducted full-scale fire test. 
	\item Attempts were then made to develop the thermal models in ABAQUS. Comparison of the thermal model results from ABAQUS with the conducted full-scale fire test on double stu LSF wall did not agree well. The plasterboard time-temperatures curves agreed reasonably well in all fire tests up to a certain time period. However, the stud time-temperature curves in double stud walls did not agree with the available experimental results. The cause was identified as the effect of natural convection within the cavity, and the same could not be simulated in the existing ABAQUS thermal model. 
	\item Even though the thermal models in ABAQUS were validated for single stud walls, the same could not be used for double stud walls. This clearly showed that the assumption of only the radiation heat transfer within the cavity to be the dominant mode of heat transfer does not hold good for double stud LSF walls. 
	\item Finally, thermal models using FDS were developed to predict the thermal response of these LSF walls in fire. This included the effects of convection within the cavity and was able to predict the time-temperature curves of the conducted full-scale fire tests with good agreement. 
	\item Obstruction removal technique at specified setpoint temperature was employed to simulate the plasterboard open up on the fire exposed side through which the sudden rise in the time-temperature curve experienced in some fire tests could be simulated with reasonable accuracy. However, the setpoint temperatures were extracted from past research study and needs further investigation to determine the suitability in other complex LSF wall configurations.
	\item The thermal model results from the developed FDS thermal models were then used in structural FE models to predict the failure times of the tested and other similar double stud, shaftliner and staggered stud LSF walls configurations. 
\end{itemize}

\subsection{Structural modelling of Complex LSF Walls}
\begin{itemize}
	\item Structural models developed initially in ABAQUS to predict the ambient temperature capacity tests resulted in good agreement in regards to the ultimate failure load predictions. The buckling modes of the studs from the ambient temperature capacity test walls were also simulated with good agreement.
	\item Based on the developed ambient temperature capacity models. sequentially coupled thermal-structural analysis was carried out using ABAQUS by importing the temperatures from the FDS thermal analysis. Through this, the structural response of the steel studs in LSF walls exposed to fire were simulated. The failure times of all the walls configurations considered for the experimental investigation were predicted with good agreement through the developed structural models. Considering the plasterboard open up in the thermal analysis has resulted in increased hot flange temperatures causing premature structural failure in comparison with non-cavity insulated double stud LSF walls. Irrespective to the position in LSF wall, cavity insulation causes detrimental effects on the fire performance of cavity insulated double stud LSF walls. 
	\item However, the developed sequentially coupled temperature displacement model suffered convergence issue in many cases and the failure modes of the studs could not be simulated well. This was attributed to the dissipated energy factor in the analysis and determining the same for every model was challenging. Further investigation on this might solve the convergence issues with the developed structural model.
\end{itemize}

\section{Future Research and Recommendations}

This research has presented with valuable results from the full-scale fire tests on complex LSF walls. Also, the developed numerical models to predict the thermal and structural performance of the complex LSF walls exhibited good agreement in most cases. However, based on the current research findings and limitations, following recommendations are worthwhile for further investigation in future. 
\begin{itemize}
	\item Although full-scale fire tests are expensive and time-consuming, these are inevitable in determining the fire performance of new LSF wall configurations. For instance, the staggered stud LSF walls with different stud spacing and arrangement is worth investigating to arrive at the optimal staggered stud wall configuration. This also holds true for double stud LSF walls.
	\item Fire test on shaftliner LSF walls was conducted under non-load bearing conditions in this research due to the limitations in the experimental set-up. Conducting full-scale fire tests on the shaftliner LSF wall with a C-H stud instead of conventional lipped channel section (LCS) under load bearing conditions can improve the understanding on the fire performance of shaftliner LSF walls.
	\item The developed FDS thermal model was small-scale in nature and the plasterboard open-up was considering by removing a portion of the obstruction. However, this approach can be conservative in some cases, especially in LSF walls where the open-up and fall-off of plasterboard is minimal. Furthermore, models to simulate joint open ups near studs in particular may improve the predictions rather than partial obstruction removal.
	\item FDS may be considered as an alternative to predict the thermal behaviour of LSF walls by considering all the modes of heat transfer, but the results from the thermal models are not visually pleasing and easy to understand. Creating thermal models using other commercially available computational fluid dynamics (CFD) software packages may deemed worthwhile considering the present limitations.
	\item The structural shell modelling with sequentially coupled temperature displacement analysis technique is considered as the best fit to determine the structural behaviour of LSF walls under fire. However, in full-scale fire tests the problem statement is fully coupled. A robust fully-coupled model may be considered as an alternative but may not be the optimal solution in all circumstances considering the computational efficiency of the fully-coupled models. 
	\item The DSM predictions were limited to the ambient temperature axial capacities in this research. This can be extended to determine the failure times under fire conditions and also investigate the suitability of the existing design equations for different LSF wall configurations.
\end{itemize}



