{\setstretch{1.5}

Cold-formed steel in the Australian construction industry is experiencing a phenomenal growth. In the residential sector light gauge steel framed (LSF) walls with single stud arrangement are commonly used as load bearing and non-load bearing walls. But in situations where higher acoustic insulation levels and load bearing capacities are required such as theatres, hospitals and town houses, complex LSF walls including double stud, shaftliner and staggered stud LSF walls are used. However, fire performance of these complex LSF walls has not been investigated adequately unlike in the case of single stud LSF walls. Fire resistance of LSF wall systems is important in the emerging mid-rise cold-formed steel construction. Many studies have been conducted in recent times to improve the acoustic, thermal (energy) and fire performance of non-load bearing LSF walls. Increasing the wall cavity depth provides superior acoustic and thermal performance characteristics of LSF walls. There is insufficient research data on the fire performance of these wall systems. The problems in determining the fire performance of these LSF walls have been identified and the investigations conducted in addressing the same are summarised below.   

Firstly, this research investigated the fire performance of complex LSF walls under standard fire conditions using full scale test panels (3 m \(\times\) 3 m) made of lipped channel studs. Ten full-scale fire tests were conducted to investigate the fire performance of load bearing and non load bearing double stud, shaftliner and staggered stud walls. Ambient capacity tests on the double and staggerd stud walls were also conducted to determine the ambient temperature axial load carrying capacities of these walls. Direct Strength Method based design capacity equations were also used to determine the ambient temperature axial compression capacities of these complex LSF wall systems and their suitability was also investigated. The fire test results of double stud, shaftliner and staggered stud LSF walls were then compared with those of single stud LSF walls with varying stud depth. The fire test results revealed the presence of a unique heat transfer mechanism, resulting in an enhanced fire performance for these complex LSF walls. The discontinuous stud arrangement within the cavity of double stud LSF walls was identified as the main contributor for the delayed heat transfer mechanism. Fire test results show that the temperatures across the wall depth decrease with increasing cavity depth. LSF walls with two layers of plasterboard lining and wider cavity resulted in increased fire resistance. This research also highlights the detrimental effects of staggered stud LSF wall arrangement causing structural failure of thin-walled steel studs. 

Cavity insulation is an integral part of the LSF wall system. However, the fire performance of the cavity insulated LSF walls are often misunderstood. Past research studies have investigated the fire performance of cavity insulated LSF walls. But, the research studies were again limited to LSF walls with single row of studs. Often, cavity insulated double and staggered stud LSF walls are used as partition walls in apartments under load bearing conditions and understanding their structural behaviour when exposed to fire becomes utmost necessity. This research study has also focused on investigating the fire performance of cavity insulated double stud LSF walls exposed to fire under load bearing conditions through full-scale fire tests. Attempts were also made to alter the position of the insulation within the cavity and investigate the effects through full-scale fire tests. This included positioning the insulation on the ambient side of the test wall and also by arranging it in a staggered pattern.

Since conducting full-scale fire tests are expensive and time consuming there arises a need to develop a robust numerical model to determine the thermal and structural performance of these complex LSF walls. Therefore attempts were made to develop numerical models in SAFIR, ABAQUS and FDS to determine the thermal performance of these complex LSF walls. The conventional thermal models developed through finite element (FE) method in ABAQUS was suitable for predicting the thermal behaviour of conventional single stud LSF walls. However, these models had many limitations such as neglecting the effects of convection within the cavity and longer computational time. Likewise the SAFIR thermal model over predicted the time-temperature curves. Also, the effects of radiation were assumed to be predominant mode of heat transfer within the cavity. Despite these limitations these models were used to validate single stud LSF wall fire tests as the effect of convection was less in these walls. However, the presence of discontinuous stud arrangements within the cavity complicates the situation making the current FE thermal models to become ineffective in predicting the thermal performance of the complex LSF walls. It was found that the complex heat transfer mechanism in these LSF walls could be better simulated with FDS. Validations of the thermal models were made against the conducted full-scale fire tests and were found to exhibit good agreement. These models were then used to conduct a detailed parametric study on the LSF wall configurations which could not be test through full-scale fire tests.

The FE structural models were created in ABAQUS to predict the ambient temperature axial compression capacities and the failure times of the complex LSF walls under fire. The developed FE structural model in ABAQUS was also able to predict the axial compression capacities and failure times of the conducted full-scale ambient temperature and fire tests. The developed numerical models resulted in reasonable agreement with the experimental results and was used to predict the thermal and structural response of the complex LSF walls with and without cavity insulation under different load ratios, thereby reducing the necessity of full-scale fire tests. The details, results and limitations of the thermal and structural models are also presented and discussed in detail.

Overall, this research has investigated in detail the fire performance of complex LSF walls which includes double stud, shaftliner and staggered stud LSF walls. The conducted experimental investigations alongside the developed numerical models have facilitated the enhancement of knowledge on the thermal and structural performance of the complex LSF walls systems, thereby providing helpful insights to the cold-formed steel industry.

}
